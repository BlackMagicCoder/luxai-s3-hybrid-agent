\documentclass{article}
\usepackage{algorithm}
\usepackage{algorithmic}
\usepackage{amsmath}
\usepackage{amssymb}

\begin{document}

\section*{Hybrid Rule-Based Deep Q-Network Agent for LUX AI}

\begin{algorithm}[H]
\caption{Overall Agent Architecture}
\begin{algorithmic}[1]
\STATE \textbf{Initialize Components:}
\STATE Create Q-network (neural network for action evaluation)
\STATE Create target network (stable copy of Q-network)
\STATE Create replay buffer (stores past experiences)
\STATE Set hyperparameters: learning rate, discount factor, exploration rates
\STATE Initialize tracking systems: relic memory, productive tiles, unit assignments
\STATE
\FOR{each game}
    \STATE Reset game-specific memory
    \FOR{each match (3 matches per game)}
        \FOR{each time step}
            \STATE Observe current state and update relic positions
            \FOR{each active unit}
                \STATE Generate random number $u$ from [0,1]
                \IF{$u <$ DQN probability}
                    \STATE Select action using DQN policy (Algorithm 3)
                \ELSE
                    \STATE Select action using rule-based policy (Algorithm 4)
                \ENDIF
                \STATE Augment with attack action if appropriate (Algorithm 5)
            \ENDFOR
            \STATE Execute all unit actions
            \STATE Observe rewards and next state
            \STATE Update productive tile statistics (Algorithm 6)
            \STATE Calculate shaped rewards for each unit (Algorithm 7)
            \STATE Store transitions in replay buffer
            \IF{training step}
                \STATE Perform network update (Algorithm 8)
            \ENDIF
        \ENDFOR
        \STATE Reset match-specific memory
    \ENDFOR
    \STATE Reset all memories for next game
\ENDFOR
\end{algorithmic}
\end{algorithm}

\begin{algorithm}[H]
\caption{Neural Network Architecture}
\begin{algorithmic}[1]
\STATE \textbf{Q-Network Structure:}
\STATE Input layer: Game state features (position, energy, distances, etc.)
\STATE Hidden layer 1: 256 neurons with ReLU activation
\STATE Hidden layer 2: 256 neurons with ReLU activation
\STATE Hidden layer 3: 256 neurons with ReLU activation
\STATE Hidden layer 4: 128 neurons with ReLU activation
\STATE Output layer: 5 neurons (Q-values for each action)
\STATE \quad Actions: 0=Stay, 1=Up, 2=Right, 3=Down, 4=Left
\STATE
\STATE \textbf{Target Network:}
\STATE Same architecture as Q-network
\STATE Updated periodically (every $N$ training steps) for learning stability
\end{algorithmic}
\end{algorithm}

\begin{algorithm}[H]
\caption{DQN Action Selection (Epsilon-Greedy)}
\begin{algorithmic}[1]
\REQUIRE Current state, Q-network, exploration rate
\STATE
\STATE Generate random number $u$ from [0,1]
\IF{$u <$ exploration rate}
    \STATE \textbf{return} random action \quad \textit{// Exploration}
\ELSE
    \STATE Compute Q-values for all actions using Q-network
    \STATE \textbf{return} action with highest Q-value \quad \textit{// Exploitation}
\ENDIF
\end{algorithmic}
\end{algorithm}

\begin{algorithm}[H]
\caption{Rule-Based Policy}
\begin{algorithmic}[1]
\REQUIRE Unit ID, time step, game state, relic positions, productive tiles
\STATE
\STATE Extract unit position, energy level, and active status
\IF{unit inactive OR energy below safety threshold}
    \STATE \textbf{return} Stay action
\ENDIF
\STATE
\STATE Determine number of active units
\STATE Calculate unit's rank among active units
\STATE Compute exploration ratio based on game phase:
\STATE \quad Early matches: higher exploration (find new relics)
\STATE \quad Late matches: lower exploration (focus on collection)
\STATE
\STATE Assign unit role based on rank and exploration ratio
\STATE
\IF{unit is COLLECTOR}
    \STATE \textbf{Priority 1: Stay on productive tiles}
    \IF{current position is productive tile}
        \STATE \textbf{return} Stay action
    \ENDIF
    \STATE
    \STATE \textbf{Priority 2: Move to known productive tiles}
    \IF{productive tiles exist}
        \STATE Find nearest productive tile
        \STATE \textbf{return} direction toward nearest productive tile
    \ENDIF
    \STATE
    \STATE \textbf{Priority 3: Systematic grid exploration}
    \IF{unit has no assignment}
        \STATE Assign unit to position in 5×5 grid around relics
    \ENDIF
    \STATE Calculate target position from assignment
    \STATE \textbf{return} direction toward assigned grid position
\ELSE
    \STATE \textit{// Unit is EXPLORER}
    \STATE \textbf{Systematic quadrant search}
    \IF{unit has no exploration target}
        \STATE Assign quadrant based on unit ID (4 quadrants total)
        \STATE Generate random target within assigned quadrant
    \ENDIF
    \STATE
    \IF{target reached OR stuck at current target}
        \STATE Generate new random target in quadrant
    \ENDIF
    \STATE \textbf{return} direction toward exploration target
\ENDIF
\end{algorithmic}
\end{algorithm}

\begin{algorithm}[H]
\caption{Attack Action Selection}
\begin{algorithmic}[1]
\REQUIRE Unit position, game state, unit energy, attack range
\STATE
\STATE Get opponent positions and ally positions
\IF{energy insufficient OR no opponents visible}
    \STATE \textbf{return} no attack
\ENDIF
\STATE
\STATE Initialize best score = $-\infty$ and best target = none
\FOR{each possible target location within attack range}
    \STATE Calculate direct hit score (enemies at exact target)
    \STATE Calculate splash score (enemies adjacent to target)
    \STATE Calculate friendly fire penalty (allies at or near target)
    \STATE Calculate distance penalty (slight preference for closer targets)
    \STATE
    \STATE Total score = direct + splash $-$ friendly fire $-$ distance
    \STATE
    \IF{score $>$ best score}
        \STATE Update best score and best target
    \ENDIF
\ENDFOR
\STATE
\IF{best score $\geq$ minimum threshold}
    \STATE \textbf{return} attack toward best target
\ELSE
    \STATE \textbf{return} no attack
\ENDIF
\end{algorithmic}
\end{algorithm}

\begin{algorithm}[H]
\caption{Productive Tile Detection}
\begin{algorithmic}[1]
\REQUIRE Current state, team points, previous points, position statistics
\STATE
\STATE Calculate points gained this step = current points $-$ previous points
\STATE Find all unit positions near known relics (within distance 3)
\STATE
\FOR{each position near relics}
    \STATE Increment occupancy counter for this position
\ENDFOR
\STATE
\IF{points gained $>$ 0 AND units are near relics}
    \STATE Distribute gained points among near-relic positions
    \FOR{each position that was occupied}
        \STATE Update total points earned at this position
        \STATE Calculate confidence = total points / times occupied
        \STATE
        \IF{confidence $>$ 0.5 AND occupied at least 3 times}
            \STATE Mark position as "productive tile"
        \ENDIF
    \ENDFOR
\ENDIF
\STATE
\STATE \textbf{Periodic cleanup (every 100 steps):}
\STATE Remove tiles from productive set if confidence drops below 0.7
\end{algorithmic}
\end{algorithm}

\begin{algorithm}[H]
\caption{Reward Shaping Function}
\begin{algorithmic}[1]
\REQUIRE State, action, team reward, productive tiles, statistics
\STATE
\STATE Initialize shaped reward = 3.0 $\times$ team reward
\STATE Extract current position and next position from states
\STATE
\STATE \textbf{Component 1: Productive Tile Bonuses}
\IF{currently on productive tile}
    \STATE Add large bonus scaled by tile confidence
    \IF{stayed on same tile (didn't move)}
        \STATE Add additional staying bonus
    \ENDIF
\ENDIF
\IF{moved to a productive tile}
    \STATE Add bonus for reaching productive tile
\ENDIF
\STATE
\STATE \textbf{Component 2: Distance Improvement}
\IF{productive tiles are known}
    \STATE Calculate distance improvement toward nearest productive tile
    \STATE Add reward proportional to improvement
\ELSE
    \STATE Calculate distance improvement toward nearest relic
    \STATE Add smaller reward proportional to improvement
\ENDIF
\STATE
\STATE \textbf{Component 3: Energy Management}
\STATE Calculate energy change
\IF{energy increased}
    \STATE Add small bonus for energy recovery
\ENDIF
\IF{energy below threshold AND not on productive tile}
    \STATE Add penalty for risky low energy state
\ENDIF
\STATE
\STATE \textbf{Component 4: Movement Encouragement}
\IF{unit moved AND not leaving productive tile}
    \STATE Add small bonus to encourage exploration
\ENDIF
\STATE
\STATE \textbf{Component 5: Survival Bonus}
\STATE Add small constant bonus for staying alive
\STATE
\STATE \textbf{Component 6: Leaving Productive Tile Penalty}
\IF{left productive tile without going to another productive tile}
    \STATE Add penalty scaled by confidence of abandoned tile
\ENDIF
\STATE
\STATE \textbf{return} shaped reward
\end{algorithmic}
\end{algorithm}

\begin{algorithm}[H]
\caption{Training Procedure (Double DQN)}
\begin{algorithmic}[1]
\REQUIRE Replay buffer, batch size, Q-network, target network
\STATE
\IF{not enough experiences in buffer}
    \STATE \textbf{return} \quad \textit{// Skip training}
\ENDIF
\STATE
\STATE Sample random batch of experiences from replay buffer
\STATE Each experience contains: (state, action, reward, next state, done flag)
\STATE
\STATE \textbf{Step 1: Compute current Q-value predictions}
\STATE Pass states through Q-network
\STATE Extract Q-values for actions that were actually taken
\STATE
\STATE \textbf{Step 2: Compute target Q-values (Double DQN method)}
\STATE Use Q-network to SELECT best actions for next states
\STATE Use target network to EVALUATE those actions
\STATE Calculate targets = reward + discount $\times$ evaluated Q-value
\STATE If episode ended, target = reward only (no future value)
\STATE
\STATE \textbf{Step 3: Calculate loss and update}
\STATE Compute mean squared error between predictions and targets
\STATE Calculate gradients of loss with respect to network parameters
\STATE Clip gradients to prevent instability
\STATE Update network parameters using Adam optimizer
\STATE
\STATE \textbf{Step 4: Decay exploration rate}
\STATE Reduce exploration rate: new rate = current rate $\times$ decay factor
\STATE Ensure rate doesn't go below minimum value
\STATE
\STATE \textbf{Step 5: Periodically update target network}
\IF{training step is multiple of update frequency}
    \STATE Copy Q-network parameters to target network
\ENDIF
\end{algorithmic}
\end{algorithm}

\section*{Key Concepts Explained}

\subsection*{Hybrid Policy}
The agent combines two decision-making strategies:
\begin{itemize}
\item \textbf{Rule-based policy:} Hand-crafted rules for reliable behavior
\item \textbf{DQN policy:} Learned through experience via neural networks
\item A probability parameter controls which policy is used
\end{itemize}

\subsection*{Q-Network}
A neural network that estimates the "quality" (Q-value) of each action in a given state. Higher Q-values indicate better expected future rewards.

\subsection*{Experience Replay}
Past experiences are stored in a buffer and randomly sampled during training. This breaks correlations between consecutive experiences and improves learning stability.

\subsection*{Target Network}
A slowly-updated copy of the Q-network used for computing learning targets. This prevents the "moving target" problem and stabilizes training.

\subsection*{Reward Shaping}
The basic team reward is augmented with additional signals to guide learning:
\begin{itemize}
\item Bonuses for productive tiles
\item Distance improvement rewards
\item Energy management signals
\item Exploration encouragement
\end{itemize}

\subsection*{Productive Tile Detection}
Statistical tracking identifies which positions near relics actually generate points. Confidence increases as more evidence accumulates.

\end{document}
